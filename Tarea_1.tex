\documentclass[12pt,legalpaper]{article}
\usepackage[utf8]{inputenc}
\usepackage[spanish]{babel}
\usepackage{amsmath}
\usepackage{amsfonts}
\usepackage{amssymb}
\usepackage{float}
\usepackage{multicol}
\usepackage{graphicx}
\usepackage{lastpage}
\usepackage{fancyhdr}
\usepackage{subcaption}
\usepackage{tikz}
\usepackage{pgf}
\usepackage{pifont}
\usetikzlibrary{shapes, positioning}
\newcommand*\circled[1]{\tikz[baseline=(char.base)]{
            \node[shape=circle,draw,inner sep=2pt] (char) {#1};}}
\fancypagestyle{plain}{
  \fancyhf{} % empty header and footer
  \renewcommand{\headrulewidth}{0pt} % ho header line
  \renewcommand{\footrulewidth}{0pt}% not footer line
  \fancyfoot[R]{Page \thepage\ of \pageref{LastPage}}% like fancy style
}
\pagestyle{plain}

\usepackage[left=2cm,right=2cm,top=2cm,bottom=2cm]{geometry}
\author{}
\title{Instituto Politécnico Nacional \\ Centro de Investigación en Computación \\ \vspace{5mm} Curso de Selección para Ingreso al Posgrado \\ Matemáticas para Ciencias de la Computación}
\date{}
\begin{document}
\setlength{\abovedisplayskip}{0pt}
\maketitle
\textbf{Nombre completo: Galván López Marcos}
\\

\textbf{Fecha de entrega: 23/05/2024}
\begin{center}
\textbf{Tarea No. 1 Funciones}
\end{center}
\vspace{0.3cm}

\circled{1} Halllar la imagen de las siguientes funciones

\begin{itemize}
\item \circled{1} $f: \mathbb{R} \rightarrow \mathbb{R}$, $f(x) = 3x + 1$
	\begin{quote}
	$$f = \lbrace (x, 3x + 1): x \in \mathbb{R} \rbrace$$
	Sea $y \in \textbf{Im}(f)$
	\begin{align}
	\Leftrightarrow & x \in \mathbb{R} \notag\\
	\Leftrightarrow & f(x) = y \notag\\
	\Leftrightarrow & 3x + 1 = y \notag\\
	\Leftrightarrow & x = \frac{y-1}{3}	\notag
	\end{align}
	
	$$\textbf{Im}(f) = \textbf{Im}(3x+1) = \{ y \in \mathbb{R} \}$$

	\end{quote}
	
\item \circled{2} $f: \mathbb{R} \rightarrow \mathbb{R}$, $f(x) = \sqrt{x^{2} + 1}$
	\begin{quote}
	$$f = \lbrace (x, \sqrt{x^{2} + 1}): x \in \mathbb{R} \rbrace$$
	Sea $y \in \textbf{Im}(f)$
	\begin{align}
	\Leftrightarrow & x \in \mathbb{R} \notag\\
	\Leftrightarrow & f(x) = y \notag\\
	\Leftrightarrow & \sqrt{x^{2} + 1} = y \notag\\
	\Leftrightarrow & x = \sqrt{y^{2} - 1}	\notag
	\end{align}
	
	$$\textbf{Im}(f) = \textbf{Im}(\sqrt{x^{2} + 1}) = \{y \in R: y \geq 1 \wedge y \leq -1 \}$$
	\end{quote}
	
\item \circled{3} $f: \mathbb{R} \rightarrow \mathbb{R}$, $f(x) = \frac{10x}{x + 5}$
	\begin{quote}
	$$f = \lbrace (x, \sqrt{x^{2} + 1}): x \in \mathbb{R} \rbrace$$
	Sea $y \in \textbf{Im}(f)$
	\begin{align}
	\Leftrightarrow & x \in \mathbb{R} \notag\\
	\Leftrightarrow & f(x) = y \notag\\
	\Leftrightarrow & \frac{10x}{x + 5} = y \notag\\
	\Leftrightarrow & x = \frac{5y}{10 - y}	\notag
	\end{align}
	
	$$\textbf{Im}(f) = \textbf{Im}(\frac{10x}{x + 5}) = \{y \in R: y \neq 10 \}$$
	\end{quote}
	
\end{itemize}

\circled{2} Da un ejemplo de una función tal que: sea

\begin{itemize}
\item \circled{i} inyectiva y, no sobre
$$f: \mathbb{N} \rightarrow \mathbb{N} \quad f(x) = 2x$$
\item \circled{ii} sobre y, no inyectiva
$$f: \mathbb{R}  \rightarrow \mathbb{R} \quad f(x) = x^{2}$$
\item \circled{iii} inyectiva y, sobre
$$f: \mathbb{R} \rightarrow \mathbb{R} \quad f(x) = 3x - 7$$
\item \circled{iv} no inectiva y, no sobre
$$f: \mathbb{R} \rightarrow \mathbb{R} \quad f(x) = \log(x) + 1$$
\end{itemize}

\end{document}